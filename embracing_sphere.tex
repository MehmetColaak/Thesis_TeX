\documentclass[a4paper, 12pt]{report}

\author{Mehmet Colak}
\title{Embracing Sphere: A Critical Approach to the Adoption
of Immersive Audio Technologies in Sound Art}
\date{01.05.2025}

\begin{document}

\maketitle
\tableofcontents

\begin{abstract}
    This thesis is about\ldots
\end{abstract}

\newpage

%   Chapter I Introduction

\chapter{Introduction}
    \section{Background, State of Art}
        \subsection{Immersion and Presence}
        \subsection{Immersive Audio Technologies}
        \subsection{Space and Audio}
    \section{Research Question and Objectives}
        \subsection{Benefits of Audio in Immersive Media}
        \subsection{Conservatism in Technology Adoption}
        \subsection{Stigmatization and Social Perceptions}
    \section{Framework and Core Concepts}
        \subsection{Abbreviatons and Key Terms}
        \subsection{Technologies and Concept Overviews}

%   Chapter II Literature Review

\chapter{Literature Review}
    \section{Evolution of Audio Technologies}
        \subsection{History in Recorded Audio}
        \subsection{Theoretical Foundations of Spatial Sound}
        \subsection{Immersive Audio Standarts and Frameworks}
        \subsection{Open-Source Contributions}
    \section{Artistic and Practical Applications}
        \subsection{Immersive Audio in Media Art}
        \subsection{Augmented and Virtual Reality}
    \section{Artwork Examples}
        \subsection{Pioneering Sound Artist and Works}
        \subsection{Cross-Disciplinary Artworks}
        \subsection{Sonic Interaction and Sound Installations}
    \section{Challenges in Adoption}
        \subsection{Economic Barriers}
        \subsection{Technical Complexity}
        \subsection{Cultural and Educational Gaps}
    \section{Future Directions}
        \subsection{Artificial Intelligence and Generative Adversarial Networks}

%   Chapter III Methodology and Case Study

\chapter{Methodology and Case Study}
    \section{Data Collection Analysis}
        \subsection{Purpose and Scope}
    \section{Artistic Experimentation}
        \subsection{A Glimpse from Tannhauser Gate}
        \subsection{Description of Proposed Artwork}
        \subsection{Human Interaction Goals}
    \section{Technical Methods and Sound Approach}
        \subsection{Screen Based Eye Tracking}
        \subsection{Generative RIR, Interactive Environments I}
        \subsection{Spatial Audio System, Interactive Environments II}
        \subsection{Sonic Environmental Narration Endpoints}
    \section{Data Collection Tools}
        \subsection{Eye Tracking Heat Maps}
        \subsection{User Unique Audio Narration Sequences}
        \subsection{Questionnaire}
    \section{Analysis Framework}
        \subsection{Self Observations}
        \subsection{Identified Barriers and Biases}
        \subsection{Feasibility}
    
%   Chapter IV Result and Discussions

\chapter{Result and Discussions}
    \section{Audience Feedback and Analysis}
        \subsection{Statistical Insights}
        \subsection{Expectations and Questionnaire Results}
    \section{Key Themes and Observations}
        \subsection{Intuitivenes of Interaction}
        \subsection{Technical Challenges and Opportunities}
    \section{Comparison with Literature}
        \subsection{Alignment with Prior Research}
        \subsection{Identified Differences}
    \section{Audience Engagement}
        \subsection{Interactive Pattern and Behaviours}
        \subsection{Role of Artistic Expression}

%   Chapter V Conclusion

\chapter{Conclusion}
    \section{Summary of Findings}
    \section{Implications for Artist and Developers}
    \section{Limitations of Research}
    \section{Final Thoughts}

% Chapter VI References

\bibliography{chapter{Bibliography}}


\end{document}