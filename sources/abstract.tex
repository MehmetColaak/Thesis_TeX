\begin{abstract}
    Immersive audio technologies, such as ambisonics and HRTF (Head-Related Transfer Function), aim to replicate spatial soundscapes, creating immersive auditory experiences. Despite their potential applications in gaming, virtual reality, cinema, and assistive tools for the visually impaired, the adoption of these technologies has faced barriers. These include technical challenges, high implementation costs, limited public awareness and the need for compatible hardware and software.

This applied work master's thesis will investigate the current state of immersive audio technologies, factors limiting its broader adoption and opportunities to accelerate innovation, particularly in enhancing storytelling.

The world that we live in is the world dominated by stereophonic and in some cases, monophonic audio formats in media. Despite technological advances, people continue to practice conservatism in music mediums, attaching technologies like vinyl records and cassette tapes even as newer platforms offer significant innovations. With the historical references of audio technology experimentations, this thesis aims to identify patterns of technological resistance and conservatism in audio technology, with a specific focus on spatial audio technologies and their limited usage in media contexts.

To explore these dynamics, the thesis will use a questionnaire after presenting sound art created using immersive audio technologies. This will help reveal how these technologies are perceived, any barriers or biases to their adoption and the overall acceptance of these immersive audio technologies.
\end{abstract}